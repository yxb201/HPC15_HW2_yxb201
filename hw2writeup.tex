\documentclass[12pt]{article}

%% FONTS
%% To get the default sans serif font in latex, uncomment following line:
 \renewcommand*\familydefault{\sfdefault}
%%
%% to get Arial font as the sans serif font, uncomment following line:
%% \renewcommand{\sfdefault}{phv} % phv is the Arial font
%%
%% to get Helvetica font as the sans serif font, uncomment following line:
% \usepackage{helvet}
\usepackage[small,bf,up]{caption}
\renewcommand{\captionfont}{\footnotesize}
\usepackage[left=1in,right=1in,top=1in,bottom=1in]{geometry}
\usepackage{graphics,epsfig,graphicx,float,subfigure,color}
\usepackage{amsmath,amssymb,amsbsy,amsfonts,amsthm}
\usepackage{url}
\usepackage{boxedminipage}
\usepackage[sf,bf,tiny]{titlesec}
 \usepackage[plainpages=false, colorlinks=true,
   citecolor=blue, filecolor=blue, linkcolor=blue,
   urlcolor=blue]{hyperref}
\usepackage{enumitem}

\newcommand{\todo}[1]{\textcolor{red}{#1}}
% see documentation for titlesec package
% \titleformat{\section}{\large \sffamily \bfseries}
\titlelabel{\thetitle.\,\,\,}


\newcommand{\bs}{\boldsymbol}
\newcommand{\alert}[1]{\textcolor{red}{#1}}
\setlength{\emergencystretch}{20pt}

\begin{document}

\begin{center}

\large \textbf{%%
High Performance Computing \\ Assignment \#2 \\ Yuan-Xun Bao \\ yxb201@nyu.edu \quad N13392943}
\end{center}

% ****************************
\section{Parallel Sample Sort}

\subsection{Strong Scalability}
To test the strong scalability of \verb|ssort|, we fix the total length of the array to be $NP = 2^{30} \approx 10^{10}$.
With this input size, \verb|ssort| scales up to 128 processors. 

\begin{center}
  \begin{tabular}{ | c | c | }
    \hline
    \# of procs & avg timing per node  \\ \hline
    16  & 18.3766s  \\ \hline
    32  & 9.2225s \\ \hline
    64  & 4.6401s  \\ \hline
    128 & 2.7381s \\ \hline
    256 & 2.2658s \\ \hline
    512 & 2.7244s \\
    \hline
  \end{tabular}
\end{center}

\subsection{Weak Scalability}

To test the weak scalability of \verb|ssort|, we fix the array length on each processor to be $N = 10^8$.
As the number of processors doubles, the computational time per node only increases slightly due to more inter-processor communications.

\begin{center}
  \begin{tabular}{ | c | c | }
    \hline
    \# of procs & avg timing per node  \\ \hline
    16  & 27.9859s  \\ \hline
    32  & 29.2918s \\ \hline
    64  & 30.2139s  \\ \hline
    128 & 30.9417s \\ \hline
    256 & 31.5131s \\ \hline
    512 & 32.7298s \\
    \hline
  \end{tabular}
\end{center}



\end{document}
